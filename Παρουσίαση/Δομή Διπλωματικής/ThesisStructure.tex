\documentclass[a4paper, 12pt, twoside]{report}

\usepackage{fontspec}
\usepackage{xunicode}
\usepackage{xltxtra}
\usepackage{xgreek}
\usepackage{csquotes}
\usepackage{placeins}

\raggedbottom
\onecolumn

\usepackage{graphicx, amsfonts, psfrag, fancyhdr, layout, subfigure}
\usepackage{multirow}
\usepackage{longtable}
\usepackage{array}

\usepackage[toc,page]{appendix}
\renewcommand{\appendixtocname}{Παραρτήματα}
\renewcommand{\appendixname}{Παράρτημα}
\renewcommand{\appendixpagename}{Παραρτήματα}

\setmainfont[Mapping=tex-text]{GFS Artemisia}

\usepackage[style=numeric, bibstyle=numeric, hyperref=true, backref=true, alldates=terse, indexing=false, backend=bibtex]{biblatex}
\addbibresource{Bibliography.bib}

\usepackage{index}
\usepackage[columns=2]{idxlayout}
\newindex{default}{idx}{ind}{Ευρετήριο}

\usepackage{hyperref}

% Set equal margins on book style
\setlength{\oddsidemargin}{53pt}
\setlength{\evensidemargin}{53pt}
\setlength{\marginparwidth}{57pt}
\setlength{\footskip}{30pt}


\begin{document}

		\paragraph{}{Στην παρούσα διπλωματική εργασία μελετάμε τη χρησιμότητα της διαγραμματικής μοντελοποίησης και δει της γλώσσας Sysml στην περιγραφή και κατασκευή μηχανοτρονικών συστημάτων. Θα προσπαθήσουμε να ελέγξουμε τις δυνατότητές της αναφορικά με τα ιδιαίτερα χαρακτηριστικά των συστημάτων αυτών. Δηλαδή θα εξετάσουμε πως μπορεί η γλώσσα αυτή να περιγράψει συστατικά όπως μηχανικά μέρη, μικροεπεξεργαστές, αισθητήρες, ενεργοποιητές αλλά και λογισμικό. Μπορεί η SysML να αναπαραστήσει επαρκώς την αλληλεπίδραση των δομικών αυτών στοιχείων; Επιπλέον, θα μελετηθεί η δυνατότητά της να μεταφέρει με πιστότητα και ευκρίνεια τα στοιχεία του μοντέλου σε όλους τους εμπλεκόμενους ανθρώπους -ηλεκτρολόγους, προγραμματιστές, χρήστες κ.λ.π..
		}
		\paragraph{}{Η δομή της προτείνεται να είναι η εξής :
			\begin{itemize}
				\item Πρόλογος
				\item Κεφάλαιο 1: Εισαγωγή στα συστήματα και στα μηχανοτρονικά συστήματα
					\begin{itemize}
						\item Σύστημα: Ιστορική αναδρομή
						\item Σύστημα: Σύγχρονη κατάσταση
						\item Μηχανοτρονικά: Ιστορική αναδρομή
						\item Μηχανοτρονικά: Σύγχρονη κατάσταση
						\item Σύνοψη
					\end{itemize}
				\item Κεφάλαιο 2 : Μοντελοποίηση συστημάτων
					\begin{itemize}
						\item Εισαγωγή
						\item Ιστορική αναδρομή
						\item Ανάλυση διαγραμματικής μοντελοποίησης
						\item Σύντομη αναφορά σε υπάρχουσες γλώσσες διαγραμματικής μοντελοποίησης
						\item Αιτιολόγηση επιλογής SysML
					\end{itemize}
				\item Κεφάλαιο 3 : Γλώσσα SysML
					\begin{itemize}
						\item Εισαγωγή
						\item Ιστορική αναδρομή
						\item Παρουσίαση διαγραμμάτων
						\item Παρουσίαση case studies
					\end{itemize}
				\item Κεφάλαιο 4 : Festo Mps
					\begin{itemize}
						\item Παρουσίαση του συστήματος
						\item Επιλογή μεθοδολογίας μοντελοποίησης
						\item Μοντελοποίηση του συστήματος
						\item Υλοποίηση του συστήματος
						\item Έλεγχος του συστήματος
					\end{itemize}
				\item Κεφάλαιο 5 : Tacos Project
				\item Κεφάλαιο 6 : Συμπεράσματα
				\item Παραρτήματα
				\item Ξενόγλωσση Βιβλιογραφία
				\item Ελληνόγλωσση Βιβλιογραφία
				\item Γλωσσάρι
				\item Συντομογραφίες
				\item Λεξικό αγγλικών όρων
				\item Ευρετήριο
			\end{itemize}
		}
		
		\paragraph{Κεφάλαιο 1: Εισαγωγή στα συστήματα και στα μηχανοτρονικά συστήματα} {Αναλύεται η έννοια του συστήματος και το επιστημονικό πεδίο των μηχανοτρονικών συστημάτων. Επιχειρείται μία ιστορική αναδρομή στην λέξη "σύστημα" και στην έννοιες που της έχουν αποδοθεί ανά τους αιώνες δίνοντας, βέβαια, μεγαλύτερη έμφαση στον 20ό αιώνα. Την περίοδο αυτή η εμφάνιση της γενικής θεωρίας συστημάτων και η ανάπτυξη πολύπλοκων εγχειρημάτων δομούν το πεδίο του μηχανικού συστημάτων. Στη συνέχεια περιγράφονται οι σύγχρονες εξελίξεις αναφορικά με την κατασκευή συστημάτων. Αντίστοιχα, το ίδιο επιχειρείται και για τα μηχανοτρονικά συστήματα. Τέλος, παρατίθεται μία σύνοψη σκιαγραφώντας τη σύνδεση του μηχανικού συστημάτων με την παρούσα διπλωματική.
		}
		\paragraph{Κεφάλαιο 2 : Μοντελοποίηση συστημάτων} {Τεκμηριώνεται ο ορισμός της μοντελοποίησης συστημάτων και η χρησιμότητα αυτής στο μηχανικό συστημάτων. Παρουσιάζονται τα βασικά χαρακτηριστικά της τεχνικής αυτής και ο τρόπος με τον οποίο χρησιμοποιείται. Στη συνέχεια παρατίθενται οι κυριότερες μεθοδολογίες μοντελοποίησης συστημάτων με μία σύντομη ανάλυση τους.Ως λογική συνέχεια του προηγουμένου κεφαλαίου παρουσιάζονται οι κυριότερες γλώσσες  μοντελοποίησης συστημάτων με τα κυριότερα στοιχεία τους τους. Επιπλέον τεκμηριώνεται η επιλογή της SysML για την εργασία αυτή.
		}
		\paragraph{Κεφάλαιο 3 : Γλώσσα SysML} {Στο κεφάλαιο αυτό επιχειρείται μία αναλυτική εισαγωγή στη SysML. Με μια σύντομη εισαγωγή αναπτύσσεται η λογική και τα δομικά στοιχεία της γλώσσας. Συμπληρωματικά, επιχειρείται ιστορική αναδρομή στη γέννηση και στην εξέλιξη της. Τέλος, παρουσιάζονται εν συντομία τα διαγράμματα τα οποία αυτή χρησιμοποιεί.
		}
		\paragraph{Κεφάλαιο 4 : Festo Mps} {Αρχικά δίνεται μια γενική, αναλυτική, λεκτική περιγραφή του συστήματος Festo MPS και των υποσυστημάτων του. Ακολούθως, επιλέγεται η μεθοδολογία που θα ακολουθηθεί για την διαγραμματική περιγραφή και κατασκευή του Festo MPS. Στη συνέχεια παρατίθεται η σχεδίαση του συστήματος με τη γλώσσα Sysml και αναπτύσσονται τα μοντέλα που κατασκευάστηκαν για το σκοπό αυτό. Επιπλέον, αναλύεται και παρουσιάζεται το στάδιο κατασκευής και εν τέλει γίνεται ο έλεγχος του συστήματος.
		}
		\paragraph{Κεφάλαιο 5 : Takos System} {Δεν έχει οριστικοποιηθεί αν θα το συμπεριλάβουμε.
		}
		\paragraph{Κεφάλαιο 6 : Συμπεράσματα} {Στο κεφάλαιο αυτό αναλύονται τα συμπεράσματα που προέκυψαν από την διπλωματική εργασία. Αναλύονται τα πλεονεκτήματα και τα μειονεκτήματα που επιβεβαιώθηκαν από την εφαρμογή της Sysml καθώς και οι διαπιστώσεις του γράφοντα από την εμπειρία της κατασκευής των συστημάτων.
		}
		\paragraph{} {Ολοκληρώνοντας, ακολουθούν τα παραστήματα των περιπτώσεων χρήσης όπου παρατίθενται τα διαγράμματα που κατασκευάστηκαν για την υλοποίησή τους και οποιαδήποτε επιπλέον πληροφορία. Ακολουθεί ελληνόγλωσση και ξενόγλωσση βιβλιογραφία, γλωσσάρι, λίστα συντομογραφιών, λεξικό αγγλικών όρων και ευρετήριο.
		}


\end{document}